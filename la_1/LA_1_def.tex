\documentclass[12pt]{article}

\usepackage[utf8]{inputenc}
\usepackage{amsmath}
\usepackage{amssymb}
\usepackage{xcolor}
\usepackage{tikz}

\begin{document}
\paragraph{\Large LA 1 Definitionen}
\begin{flushleft}
    Diese Zusammenfassung setzt Grundwissen voraus:
    \begin{itemize}
        \item Relationen
        \item Abbildungen
        \item Mengen
        \item Aussagenlogik
        \item Rechenaxiome
    \end{itemize}
\end{flushleft}
\vspace{0.5cm}
\subparagraph{\large Aequivalenzrelationen}
\normalsize
\begin{flushleft}
    Eine Aequivalenzrelation ist eine Relation $\sim$ auf die Menge $M$ mit folgenden Eigenschaften:
    \begin{itemize}
        \item \textbf{Reflexiv}: $\forall x \in M: x \sim x$.
        \item \textbf{Symmetrie}: $\forall x,y \in M: x \sim y \Rightarrow y \sim x$.
        \item \textbf{Transitiv}: $\forall x,y,z \in M: x \sim y \land y \sim z \Rightarrow x \sim z$.
    \end{itemize}
    Ein Beispiel fuer eine Aequivalenzrelation ist $=$ auf die Menge $\mathbb{Z}$.
    \vspace{1cm}
    Es sei $\sim$ eine Aequivalenzrelation auf die Menge $M$. Dann heisst fuer $x \in M$ die Teilmenge
    \begin{align*}
        [x]_{\sim} := \{y \in M | x \sim y\} \subseteq M
    \end{align*}
    die \textbf{Aequivalenzklasse} von $x$ (bezueglich $\sim$).
\end{flushleft}
\vspace{1cm}
\subparagraph{\large Gruppen}
\normalsize
\begin{flushleft}
    Es sei $M$ eine Menge und $+$ eine Verknuepfungen auf $M$. Dann heisst das Paar $(M,+)$ eine $Gruppe$, wenn folgende Bedingungen gelten:
    \begin{itemize}
        \item Assoziativ: $\forall x,y,z \in M: (x + y) + z = x + (y + z)$.
        \item neutrales Element: $\exists e \in M; \forall x \in M: x + e = e + x = x$.
        \item inverses Element: $\forall x \in M; \exists y \in M: x + y = y + x = e$.
    \end{itemize}
    Eine besondere und wichtig Art der Gruppen sind die abelsche Gruppen (kommutative Gruppen).
    Damit eine Gruppe abelsch ist, muss die Verknuepfungen $+$ kommutativ sein.
\end{flushleft}
\vspace{1cm}
\subparagraph{\large Koerper}
\normalsize
\begin{flushleft}
    Damit eine Menge $K$ mit den zwei Verknuepfungen $+$ und $\cdot$ ein kommutativer Koerper (Wird auch einfach nur Koerper gennant) ist, muessen folgende Aussagen gelten:
    \begin{itemize}
        \item \textbf{Gesetze der Addition}
        \begin{itemize}
            \item Assoziativitaet:
                $(x + y) + z = x + (y + z).$
            \item Eindeutiges neutrales Element:
                $0 + x = x.$
            \item Eindeutiges inverses Element: 
                $x + (-x) = 0.$
            \item Kommutativaet:
                $x + y = y + x.$
        \end{itemize}
        \item \textbf{Gesetze der Multiplikation}
        \begin{itemize}
            \item Assoziativitaet:
                $x \cdot (y \cdot y) = (x \cdot y) \cdot z.$
            \item Eindeutiges neutrales Element:
                $1 \cdot x = x \cdot 1 = x.$
            \item Eindeutiges inverses Element:
                $x \cdot x^{-1} = 1 = x^{-1} \cdot x.$
            \item Kommutativaet:
                $x \cdot y = y \cdot x.$
        \end{itemize}
        \item \textbf{Distributivgesetz}
            $x \cdot (y + z) = x \cdot y + x \cdot z.$
        \item Gesetze die fuer \textbf{jeden} Koerper gelten:
        \begin{itemize}
            \item Multiplikation mit 0:
                $\forall x \in K: x \cdot 0 = 0$
            \item Nullteilerfreiheit:
                $x,y \in K, x \not = 0, y \not = 0 \Rightarrow x \cdot y \not = 0.$
        \end{itemize}
    \end{itemize}
    Ausserdem muessen die beiden Verknuepfungen $+$ und $\cdot$ Verknuepfungen auf $K$ sein. Also eine binaere Abbildung $K \times K \rightarrow K$.
    Entscheiden ist dabei, dass das Bild wieder ein Element von $K$ seien muss; diese Forderung bezeichnet man auch als \textbf{Abgeschlossenheit}.
\end{flushleft}
\vspace{1cm}
\subparagraph{\large Abstrakte Vektorraeume}
\normalsize
\begin{flushleft}
    Ein Vektorraum wird immer ueber einen Koerper definiert! (Koerper sind zB. $\mathbb{Q}$, $\mathbb{R}$ und $\mathbb{C}$).
    Ein Vektorraum $V$ ueber Koerper $\mathbb{K}$ hat folgende Eigenschaften:
    \begin{itemize}
        \item \textbf{Vektoraddition}:
        \begin{itemize}
            \item (V1) Assoziativitaet $\forall u,v,w \in V: (u + v) + w = u + (v + w).$
            \item (V2) Eindeutiges neutrales Element $\exists ! o \in V; \forall v \in V: o + v = v + o = v.$
            \item (V3) Existenz eines inversen Elements $\forall v \in V; \exists !(-v)\in V: v + (-v) = (-v) + v = o.$
            \item (V4) Kommutativaet $\forall v,w \in V: v + w = w + v.$
        \end{itemize}
        \item \textbf{Skalarmultiplikation}
        \begin{itemize}
            \item (S1) Assoziativitaet $\forall \lambda , \mu \in \mathbb{K}; \forall v \in V: \lambda(\mu v) = (\lambda \mu)(v).$
            \item (S2) Wirkung des neutralen Elements $\forall v \in V: 1 \cdot v = v.$
        \end{itemize}
        \item \textbf{Vektoraddition und Skalarmultiplikation}
        \begin{itemize}
            \item (D1) 1. Distributivgesetz $\forall \lambda \in \mathbb{K}; \forall v,w \in V: \lambda(v + w) = (\lambda v) + (\lambda w).$
            \item (D2) 2. Distributivgesetz $\forall \lambda , \mu \in \mathbb{K}; \forall v \in V: (\lambda + \mu)v = (\lambda v) + (\mu v).$
        \end{itemize}
    \end{itemize}
    Sei $\mathbb{K}$ ein Koerper und $V$ eine Menge. Sind \begin{align}
        +: V \times V \rightarrow V, \quad (x,y) \mapsto x + y \quad \text{und} \quad \cdot: \mathbb{K} \times V \to V, \quad (\lambda,x) \mapsto \lambda x
    \end{align}
    zwei Abbildungen, die den Eigenschaften (V1-4) (S1) (S2) und (D1) (D2) genuegen, so heisst das Triple $(V,+,\cdot)$ ein $\mathbb{K}$-Vektorraum.
\end{flushleft}
\end{document}