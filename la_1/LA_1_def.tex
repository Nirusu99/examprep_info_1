\documentclass[12pt]{article}

\usepackage[utf8]{inputenc}
\usepackage{amsmath}
\usepackage{amssymb}
\usepackage{xcolor}
\usepackage{tikz}

\begin{document}
\paragraph{\Large LA 1 Definitionen}
\begin{flushleft}
    Diese Zusammenfassung setzt Grundwissen voraus:
    \begin{itemize}
        \item Abbildungen
        \item Mengen
        \item Aussagenlogik
        \item Rechenaxiome
    \end{itemize}
\end{flushleft}
\vspace{1cm}
\subparagraph{\large Koerper}
\normalsize
\begin{flushleft}
    Damit eine Menge $K$ mit den zwei Verknuepfungen $+$ und $\cdot$ ein kommutativer Koerper (Wird auch einfach nur Koerper gennant) ist, muessen folgende Aussagen gelten:
    \begin{itemize}
        \item Gesetze der Addition
        \begin{itemize}
            \item Assoziativitaet: \begin{align*}
                (x + y) + z = x + (y + z).
            \end{align*}
            \item Eindeutiges neutrales Element: \begin{align*}
                0 + x = x.
            \end{align*}
            \item Eindeutiges inverses Element: \begin{align*}
                x + (-x) = 0.
            \end{align*}
            \item Kommutativaet: \begin{align*}
                x + y = y + x.
            \end{align*}
        \end{itemize}
        \item Gesetze der Multiplikation
        \begin{itemize}
            \item Assoziativitaet: \begin{align*}
                x \cdot (y \cdot y) = (x \cdot y) \cdot z.
            \end{align*}
            \item Eindeutiges neutrales Element: \begin{align*}
                1 \cdot x = x \cdot 1 = x.
            \end{align*}
            \item Eindeutiges inverses Element: \begin{align*}
                x \cdot x^{-1} = 1 = x^{-1} \cdot x.
            \end{align*}
            \item Kommutativaet: \begin{align*}
                x \cdot y = y \cdot x.
            \end{align*}
        \end{itemize}
        \item Distributivgesetz
        \begin{align*}
            x \cdot (y + z) = x \cdot y + x \cdot z.
        \end{align*}
        \item Gesetze die fuer \textbf{jeden} Koerper gelten:
        \begin{itemize}
            \item Multiplikation mit 0: \begin{align*}
                \forall x \in K: x \cdot 0 = 0
            \end{align*}
            \item Nullteilerfreiheit: \begin{align*}
                x,y \in K, x \not = 0, y \not = 0 \Rightarrow x \cdot y \not = 0.
            \end{align*}
        \end{itemize}
    \end{itemize}
    Ausserdem muessen die beiden Verknuepfungen $+$ und $\cdot$ Verknuepfungen auf $K$ sein. Also eine binaere Abbildung $K \times K \rightarrow K$.
    Entscheiden ist dabei, dass das Bild wieder ein Element von $K$ seien muss; diese Forderung bezeichnet man auch als \textbf{Abgeschlossenheit}.
\end{flushleft}
\end{document}