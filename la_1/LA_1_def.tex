\documentclass[12pt]{article}

\usepackage[utf8]{inputenc}
\usepackage{amsmath}
\usepackage{amssymb}
\usepackage{xcolor}
\usepackage{tikz}

\begin{document}
\paragraph{\Large LA 1 Definitionen}
\begin{flushleft}
    Diese Zusammenfassung setzt Grundwissen voraus:
    \begin{itemize}
        \item Abbildungen
        \item Mengen
        \item Aussagenlogik
        \item Rechenaxiome
    \end{itemize}
\end{flushleft}
\vspace{1cm}
\subparagraph{\large Koerper}
\normalsize
\begin{flushleft}
    Damit eine Menge $K$ mit den zwei Verknuepfungen $+$ und $\cdot$ ein kommutativer Koerper (Wird auch einfach nur Koerper gennant) ist, muessen folgende Aussagen gelten:
    \begin{itemize}
        \item \textbf{Gesetze der Addition}
        \begin{itemize}
            \item Assoziativitaet: \begin{align*}
                (x + y) + z = x + (y + z).
            \end{align*}
            \item Eindeutiges neutrales Element: \begin{align*}
                0 + x = x.
            \end{align*}
            \item Eindeutiges inverses Element: \begin{align*}
                x + (-x) = 0.
            \end{align*}
            \item Kommutativaet: \begin{align*}
                x + y = y + x.
            \end{align*}
        \end{itemize}
        \item \textbf{Gesetze der Multiplikation}
        \begin{itemize}
            \item Assoziativitaet: \begin{align*}
                x \cdot (y \cdot y) = (x \cdot y) \cdot z.
            \end{align*}
            \item Eindeutiges neutrales Element: \begin{align*}
                1 \cdot x = x \cdot 1 = x.
            \end{align*}
            \item Eindeutiges inverses Element: \begin{align*}
                x \cdot x^{-1} = 1 = x^{-1} \cdot x.
            \end{align*}
            \item Kommutativaet: \begin{align*}
                x \cdot y = y \cdot x.
            \end{align*}
        \end{itemize}
        \item \textbf{Distributivgesetz}
        \begin{align*}
            x \cdot (y + z) = x \cdot y + x \cdot z.
        \end{align*}
        \item Gesetze die fuer \textbf{jeden} Koerper gelten:
        \begin{itemize}
            \item Multiplikation mit 0: \begin{align*}
                \forall x \in K: x \cdot 0 = 0
            \end{align*}
            \item Nullteilerfreiheit: \begin{align*}
                x,y \in K, x \not = 0, y \not = 0 \Rightarrow x \cdot y \not = 0.
            \end{align*}
        \end{itemize}
    \end{itemize}
    Ausserdem muessen die beiden Verknuepfungen $+$ und $\cdot$ Verknuepfungen auf $K$ sein. Also eine binaere Abbildung $K \times K \rightarrow K$.
    Entscheiden ist dabei, dass das Bild wieder ein Element von $K$ seien muss; diese Forderung bezeichnet man auch als \textbf{Abgeschlossenheit}.
\end{flushleft}
\vspace{1cm}
\subparagraph{\large Abstrakte Vektorraeume}
\normalsize
\begin{flushleft}
    Ein Vektorraum wird immer ueber einen Koerper definiert! (Koerper sind zB. $\mathbb{Q}$, $\mathbb{R}$ und $\mathbb{C}$).
    Ein Vektorraum $V$ ueber Koerper $\mathbb{K}$ hat folgende Eigenschaften:
    \begin{itemize}
        \item \textbf{Vektroaddition}:
        \begin{itemize}
            \item (V1) Assoziativitaet $\forall u,v,w \in V: (u + v) + w = u + (v + w).$
            \item (V2) Eindeutiges neutrales Element $\exists ! o \in V; \forall v \in V: o + v = v + o = v.$
            \item (V3) Existenz eines inversen Elements $\forall v \in V; \exists !(-v)\in V: v + (-v) = (-v) + v = o.$
            \item (V4) Kommutativaet $\forall v,w \in V: v + w = w + v.$
        \end{itemize}
        \item \textbf{Skalarmultiplikation}
        \begin{itemize}
            \item (S1) Assoziativitaet $\forall \lambda , \mu \in \mathbb{K}; \forall v \in V: \lambda(\mu v) = (\lambda \mu)(v).$
            \item (S2) Wirkung des neutralen Elements $\forall v \ in V: 1 \cdot v = v.$
        \end{itemize}
        \item \textbf{Vektoraddition und Skalarmultiplikation}
        \begin{itemize}
            \item (D1) 1. Distributivgesetz $\forall \lambda \in \mathbb{K}; \forall v,w \in V: \lambda(v + w) = (\lambda v) + (\lambda w).$
            \item (D2) 2. Distributivgesetz $\forall \lambda , \mu \in \mathbb{K}; \forall v \in V: (\lambda + \mu)v = (\lambda v) + (\mu v).$
        \end{itemize}
    \end{itemize}
    Wenn nun $\mathbb{K} := \mathbb{R}$ und $V$ eine Menge sind. Und \begin{align}
        +: V \times V \rightarrow V, \quad (x,y) \mapsto x + y \quad \text{und} \quad \cdot: \mathbb{K} \times V \to V, \quad (\lambda,x) \mapsto \lambda x
    \end{align}
    zwei Abbildungen sind, die den Eigenschaften (V1-4) (S1) (S2) und (D1) (D2) genuegen, so heisst das Triple $(V,+,\cdot)$ ein $\mathbb{R}$-Vektorraum.
\end{flushleft}
\end{document}