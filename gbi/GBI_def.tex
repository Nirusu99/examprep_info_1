\documentclass[12pt]{article}

\usepackage[utf8]{inputenc}
\usepackage{amsmath}
\usepackage{amssymb}

\begin{document}
\paragraph{GBI Definitionen}
\subparagraph{RegEx}
\begin{flushleft}
    Wissenswertes:
    \begin{itemize}
        \item Hilfssymbole $:= \{|,(,),\ast,\emptyset\}$
        \item "$\ast$ vor $\cdot$ (Konkatenation)"
        \item "$\cdot$ vor Strich "|" (Oder)
        \item $\langle R \rangle$ ist die formale Sprache ist, welche mit $R$ gebildet werden kann
        \item $\langle \emptyset \rangle = \{\}$
        \item $\langle R_1| R_2 \rangle = \langle R_1 \rangle \cup \langle R_2 \rangle$
        \item $\langle R_1 \cdot R_2 \rangle =\langle R_1 \rangle \cdot \langle R_2 \rangle$
        \item $\langle R \ast \rangle = \langle R \rangle ^\ast$
        \item Es gibt kein $R+$ sondern $RR\ast$ Bsp.: Statt $(ab)+ \text{ einfach } ab(ab)\ast$
    \end{itemize}
    Bsp.: \\
    $R = a|b$ dann ist:
    \begin{align*}
        \langle R \rangle = \langle a | b \rangle = \langle a \rangle \cup \langle b \rangle = \{a\} \cup \{b\} = \{a,b\}
    \end{align*}
    $R = (a|b)\ast$ dann ist:
    \begin{align*}
        \langle R \rangle = \langle (a|b)\ast \rangle = \langle a | b \rangle^\ast = \{a,b\}^\ast
    \end{align*}
    $R = (a\ast b\ast)\ast$ dann ist:
    \begin{align*}
        \langle R \rangle &= \langle (a\ast b\ast)\ast \rangle = \langle a\ast b\ast\rangle^\ast \\
        &= (\langle a\ast\rangle\langle b\ast\rangle)^\ast = (\langle a\rangle^\ast\langle b\rangle^\ast)^\ast = (\{a\}^\ast\{b\}^\ast)^\ast \\
        &= \{a,b\}^\ast
    \end{align*}
\end{flushleft}
\end{document}